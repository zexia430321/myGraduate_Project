% !Mode:: "TeX:UTF-8"
\addcontentsline{toc}{chapter}{致\quad 谢} %添加到目录中
\chapter*{致\quad 谢}

%湖南大学学位论文~\LaTeX~模板主要参考以下内容:
%\begin{itemize}
%  \item 天津大学~TJUThesis~硕博士学位论文模板
% \end{itemize}

%感谢~ChinaTeX~大神的无私帮助。
%
%谨将此论文模板,献给我们最爱的母校:湖南大学。
%
%\vspace*{1cm}
%本论文的工作是在我的导师[XXXX...] 教授的悉心指导下完成的,[XXXX...]
%
%教授严谨的治学态度和科学的工作方法给了我极大的帮助和影响。在此衷心感谢三年来[XXXX...] 老师对我的关心和指导。
%
%[XXXX...] 教授悉心指导我们完成了实验室的科研工作,在学习上和生活上都给予了我很大的关心和帮助,在此向[XXXX...] 老师表示衷心的谢意。
%
%[XXXX...] 教授对于我的科研工作和论文都提出了许多的宝贵意见,在此表示衷心的感谢。
%
%在实验室工作及撰写论文期间,[XXXX...] 、[XXXX...] 等同学对我论文中的[XXXX...] 研究工作给予了热情帮助,在此向他们表达我的感激之情。
%
%另外也感谢家人[XXXX...] ,他们的理解和支持使我能够在学校专心完成我的学业。
如果说人生是一首优美的乐曲,那么湖大的日子则是其中一个美妙的音符。三年前我成功考上了心心念念的湖大,一切都是那么新鲜,由于我是个跨专业的考生,在研究生期间我加倍努力,打实专业基础,在此期间由于结识了一批优秀的同学,让我对自己的人生有了规划与追求。这一路走来,有欢声笑语,也有挫败失落。三年的研究生时光即将结束,感谢陪我一起走过这段岁月的人儿,愿你们身体健康事事顺心,也同样希望你们的生活中,因为有我的存在可以带给你们一些欢乐与帮助。

首先,由衷地感谢我的导师xxx对我的悉心指导和多方关怀!老师对我的帮助不仅仅是学业上的指导,更是设计到了生活中的方方面面,老师是一个很开明的人,对于学生想做的事情只要对人生、学业有益,均会全力支持。老师是个很睿智的人,对于生活中遇到的困难,均会加以开导,每次与老师谈话完毕总是豁然开朗。同时老师是一个很严谨的人,对待工作一丝不苟,以学生学业为重,为了提升教学质量,经常想各种方法来激起学生的学习积极性与主动性。三年前,考研初试刚刚结束,我就联系了老师,知道我是跨考生后老师并没有嫌弃我基础薄弱,反而悉心指导我该看什么书籍,该朝哪方面努力,在此由衷感谢导师这三年对我的帮助与关照,您的悉心教导足矣让我受用终身。

感谢我的校外导师xxx,每次遇到不会解决的问题时,都会很耐心地给我解答。特别是择业期间,你给我提供了很多宝贵意见,结合我自身实际情况给我分析了各个企业的优劣,最终使我获得了心仪的工作。祝愿文吉刚老师在今后的每一天身体健康、事事顺心。

感谢实验室的老师以及各位同学,感谢xxx老师、xxx老师、xxx博士、xxx博士给我的指导与帮助。感谢师弟xxx,遇到问题时和你探讨总能激发灵感。感谢幽默室友xxx,因为你让我的生活变得十分有趣。感谢大佬室友xxx,是你让我了解到自己离优秀的距离还相当遥远。感谢好朋友xxx,每次和你交流我都能获得灵感。还要特别感谢xxx同学对我的帮助。

尤其要感谢我的母亲,谢谢你一直以来的支持与鼓励,教我明辨是非,教我如何做一个大写的人。感谢我的父亲,感谢您为家庭的默默付出,任劳任怨,因为您我才有机会顺利完成学业,拥有更精彩的人生。在以后的工作与生活中,我会带着你们的期望继续前行,愿你们身体健康,事事顺心。

最后,感谢参与盲审以及答辩的各位专家和老师,谢谢你们提出的宝贵意见。












