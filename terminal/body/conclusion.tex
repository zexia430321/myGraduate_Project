% !Mode:: "TeX:UTF-8"

\addcontentsline{toc}{chapter}{结论与展望} %添加到目录中
\chapter*{结论与展望}
随着科学技术的进步,超大规模集成电路的发展速度日新月异。芯片集成度以及复杂度的急剧上升增加了制造合格芯片的困难。为了确保芯片的故障覆盖率达标,需要大量的测试数据对其进行检测。庞大的测试数据增加了硬件代价以及测试时间,一种卓有成效的方法就是对测试数据进行压缩。近年来有学者提出了一种拆分压缩技术,此技术将原测试集拆分成为主分量集(在本文中由基向量生成)以及残分量集,进一部提高压缩率。主分量的选取直接影响最终的压缩率,本文在拆分压缩的基础上,对如何生成基向量展开了研究。

第一章绪论,介绍了集成电路发展的背景以及电路测试的相关先修知识,同时就国内外研究现状进行分析,最后总结本文的组织结构。第二章主要就常用的压缩方法做了详细地介绍。第一部分讲解了编码压缩技术,包括游程编码、字典编码以及统计编码。第二部分就基于线性解压缩和基于广播扫描两种非编码压缩方法做出了相应的分析与阐述,最后介绍了拆分压缩技术和哈达码变换,并对哈达码变换的优缺点进行了分析。本文就是在拆分压缩技术的基础上通过挑选出合适的基向量进行数据压缩。

本文针对测试集自身特性,在拆分压缩技术中,针对基向量的生成算法提出了两个创新点主要有:

(1)采用预填充的策略对测试集进行处理。此方法首先对测试数据进行预填充,填充的方式有直接填充和策略填充两种,填充的目的是使测试集在当前编码规则下的压缩率更高,填充完毕后的测试集不存在无关位,然后以向量间距离最大最原则选取需要的基向量。由于本文使用的是拆分压缩技术,当原测试及中包含的码字0较多会有助于压缩率的提升,建议直接将无关位填充为码字0。实验结果表明,使用预填充的方式,RL-Huff编码的压缩率可达74.32\%,相比与对测试集进行直接编码,压缩率提高了11.75\%。通过和大多数压缩方法进行对比表明,对于大多数基准电路,使用本方法均可以获得较好的效果。

(2)提出了一种基于kmeans++聚类算法结合测试集生成基向量的方法。该方法通过步骤(1)消除原测试集中的无关位,然后对已填充的测试向量进行聚类,即相似的列向量我们将其归为一类,随后取当前聚类列向量每一位的均值作为聚类中心向量,并以每一个聚类中心向量为基准,将测试集中的每一个列向量与以获得的聚类中心向量进行对比,若欧几里得距离最小则归为一类,如此反复迭代,当聚类基本不再发生变化时,最终确定的聚类中心向量即为我们所求的基向量。实验结果表明,通过使用kmeans++算法生成的基向量来进行压缩, RL-Huff编码的平均压缩率可达76.30\%,与对测试集进行直接编码压缩相比平均压缩率提高了13.73\%,与哈达码相比,压缩率提高了4.45\%。同时本人使用此方法对大电路进行了测试,在FDR编码编码方式下,比对测试集直接压缩所获取的缩率高 6.06\%。由于基向量的选取的个数会直接影响最终的压缩率,为了更好的反映两者的对应关系,本人通过选取不同个数的基向量,计算出相应的压缩率,并绘画出其相应的折线图。

(3)将kmeans++聚类算法结合单轮位翻转算法进一步提高压缩率。此方法的主要思想是,先对原测试集进行预填充,利用kmeans++聚类算法找出所需的基向量并生成主分量集,然后对主分量集进行故障模拟检测出电路的部分故障,然后对原测试集也进行故障模拟,将主分量集能检测出故障对应的确定位转化成为无关位,并生成新的测试集。最后将新的原测试集与主分量集异或,得到新的残分量集,提升压缩率。本人基于多轮位翻转算法提出一种保留原主分量集合的单轮位翻转算法,使得在进行故障模拟时,所需要的硬件代价更小。结果表明使用位翻转算法结合kmeans++算法可以将平均压缩率相比于步骤(2)提高了7\%。

针对在拆分压缩技术中的基向量的生成算法研究,文中提出的三种方法虽然在取得了一定的成果,但仍然有不足和和厄待改进之处。现总结如下:

(1)本文在第三章使用采用预填充的策略对测试集进行处理,并且其取得了较好的效果,但是主要因为是因为基向量选取方式所导致的,在预填充的测试集中,如果选举基向量的策略为随机选取,而不是使用向量间距离最大原则为依据选取会对压缩率产生较大的影响,实验表明使用随机选取方式所达到的压缩率,与使用哈达码变换所达到的压缩率十分接近。虽然本文使用了据间距最大原则,但是第一列基向量的选取依旧是随机的,初始基向量的选取对实验结果的影响较为明显,因此预填充之后,对于基向量的选择还可以继续优化,减少随机选取带来的误差。

(2)本文第四章提出了一种基于kmeans++聚类算法结合测试集生成基向量的方法。该方法的基本思想是对已填充的测试向量进行聚类,以每一个聚类中心向量为基准,将测试集中的每一个列向量与以获得的聚类中心向量进行对比,若欧几里得距离最小则归为一类,如此反复迭代,当聚类基本不再发生变化时,基向量获取结束。由实验结果可知此方法能获取较高的压缩增益,但也有缺陷,第一kmeans++算法是一个以距离为基准的算法,简单易用,其缺点是无法确定聚类数目,需要事先设定,其次在聚类时初始基向量的选取也是随机的对实验结果也会存在一定的影响。综上所述,在聚类算法的选择上还有很大的空间,可能存在某种聚类算法可以直接确定原测试集聚类个数并且硬件代价也相对较小。

(3)本文第五章通过将kmeans++聚类算法结合位翻转算法来提高压缩率,此方法的主要思想是:对主分量集进行故障模拟检测出电路的部分故障,然将原测试集部分确定位转化为无关位,从而提升压缩率。虽说针对所有基准电路,此方法提升了较高的压缩率,但是本文中提及的翻转算法只翻转了一轮,并且翻转时使用的是贪婪算法,可能无法到达全局最优的效果,因此对翻转算法进一步优化增加可翻转的确定位,是一个可以研究的突破点,比如根据主分量集合的确定位情况设计相应的翻转算法,从而提高残分量的压缩率。
