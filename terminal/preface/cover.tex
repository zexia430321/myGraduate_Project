% !Mode:: "TeX:UTF-8"

\chnunumer{10532}
\chnuname{湖南大学}
\cclassnumber{TP391}
\cnumber{XXX}
\csecret{普通}
\cmajor{SoC测试与设计}
\cheading{工程硕士学位论文}      % 设置正文的页眉,以及自己的学位级别
\ctitle{拆分压缩中基向量的生成算法研究}  %封面用论文标题,自己可手动断行
\etitle{Research on Generation Algorithm of Base Vector in Split Compression}
\caffil{信息科学与工程学院} %学院名称
\csubjecttitle{学科专业}
\csubject{计算机技术}   %专业
\cauthortitle{研究生}     % 学位
\cauthor{XXX}   %学生姓名
\ename{XXX}
\cbe{B.E.~(Hunan Agricultural University)~2017}
%\cms{M.S.~(Hunan University)2010}
\cdegree{thesis}
\cclass{Master of engineering}
\emajor{Computer Technology}
\ehnu{Hunan~University}
\esupervisor{XXX}
\csupervisortitle{指导教师}
\csupervisor{XXX} %导师姓名
\elevel{Professor} %导师职称
\cchair{XXX}
\ddate{2020年4月30日}
\edate{May,~2020}
\untitle{湖~~南~~大~~学}
\declaretitle{学位论文原创性声明}
\declarecontent{
本人郑重声明:所呈交的论文是本人在导师的指导下独立进行研究所取得的研究成果。除了文中特别加以标注引用的内容外,本论文不包含任何其他个人或集体已经发表或撰写的成果作品。对本文的研究做出重要贡献的个人和集体,均已在文中以明确方式标明。本人完全意识到本声明的法律后果由本人承担。
}
\authorizationtitle{学位论文版权使用授权书}
\authorizationcontent{
本学位论文作者完全了解学校有关保留、使用学位论文的规定,同意学校保留并向国家有关部门或机构送交论文的复印件和电子版,允许论文被查阅和借阅。本人授权湖南大学可以将本学位论文的全部或部分内容编入有关数据库进行检索,可以采用影印、缩印或扫描等复制手段保存和汇编本学位论文。
}
\authorizationadd{本学位论文属于}
\authorsigncap{作者签名:}
\supervisorsigncap{导师签名:}
\signdatecap{签字日期:}


%\cdate{\CJKdigits{\the\year} 年\CJKnumber{\the\month} 月 \CJKnumber{\the\day} 日}
% 如需改成二零一二年四月二十五日的格式,可以直接输入,即如下所示
% \cdate{二零一二年四月二十五日}
\cdate{2020年4月24日} % 此日期显示格式为阿拉伯数字 如2012年4月25日
\cabstract{
在20世纪90年代,一个较复杂的芯片包含几十万至几百万个晶体管,而现在一个芯片可以包含上百亿个晶体管。芯片日益增加的集成度和复杂度直接提高了生产故障芯片的概率。为了确保芯片的故障覆盖率达标,需要大量的测试数据对其进行检测。庞大的测试数据增加了硬件代价和测试时间,通过压缩测试数据能大大降低被测时间并节约硬件成本。本文在拆分压缩技术的基础上,对如何生成基向量展开了研究,主要做了以下三个方面的工作:

(1)提出了一种预填充的策略对测试集进行处理。此方法通过预先对测试集进行填充消除所包含的无关位。填充方式可分为直接填充和策略填充两种,直接填充指是将原测试集中的无关位直接填充为码字0或者码字1,策略填充则是将无关位按照有助于提高特定编码压缩率的方式进行填充。对测试集填充后,本文以向量间距离最大为原则选取所需基向量。实验结果表明,通过采取预填充的压缩方法,RL-Huff 编码的压缩率可达74.32\%,相比于对测试集直接进行编码,压缩率提高了11.75\%,相比于哈达码变换,压缩率提高了2.47\%。

(2)提出了一种基于kmeans++聚类算法结合测试集生成基向量的方法。该方法首先以距离最大为原则挑选出测试集的初始聚类中心,然后计算出测试集中每一个列向量与聚类中心之间的欧几里得距离,以距离最小为原则将列向量重新划分到不同的聚类,并计算出新的聚类中心,如此反复迭代,当聚类中的列向量基本不再发生变化时,所得的聚类中心即为本文所求基向量。实验结果表明,通过使用kmeans++ 聚类算法结合测试集生成基向量的方法可使RL-Huff 编码的平均压缩率提高至76.30\%,与对测试集直接进行编码压缩相比,平均压缩率提高了13.73\%,与哈达码变换相比,平均压缩率提高了4.45\%。为了进一步验证方法的可靠性,本人对b15、b17、b20、b21 等大电路进行了相关实验,在FDR 编码方式下,此方法所获取的压缩率比对测试集直接进行压缩高 6.06\%。

(3)将kmeans++聚类算法结合位翻转算法进一步提高压缩率。此方法的主要思想是:首先对主分量集进行故障模拟检测出电路的部分故障$X$,假设原测试集的故障覆盖率为1,则原测试集只需检测出1-$X$这部分故障即可。为了避免相同的故障被反复检测,本文在不影响故障覆盖率的情况下将原测试中部分确定位翻转为无关位,然后将位翻转之后的测试集与主分量集进行异或获取残分量集,从而提升压缩率。实验结果表明通过使用kmeans++ 聚类算法结合位翻转算法可以将平均压缩率在(2)的基础上提高7\%。

}
\ckeywords{拆分压缩;基向量;主分量集;kmeans++ 算法;测试数据压缩}
\eabstract{
In the 1990s, a more complex chip contained hundreds of thousands to millions of transistors, but now a chip can contain tens of billions of transistors. The increasing integration and complexity of chips directly increase the probability of producing failed chips. In order to ensure that the tested chip has 100\% fault coverage, a large amount of test data is required to detect it. The huge test data increases the hardware cost and test application time. Compressing the test data can greatly reduce the test time and save the hardware storage overhead. In this paper, based on the split compression technology, we study how to generate basis vectors, and mainly do the following three aspects of work:

(1)A pre-populated strategy is proposed to process the test set. This method aims to eliminate extraneous bits in the test set and fill the test set in advance. The specific stuffing can be divided into direct stuffing and strategy stuffing. Direct stuffing means replacing the irrelevant bits in the original test set with codeword 0. Strategy stuffing is to fill the irrelevant bits in a way that contributes to a particular encoding compression. After filling the test set, this paper selects the required base vector based on the principle of the largest distance between vectors. The experimental results show that, using the pre-filling method, the compression rate of RL-Huff coding can reach 74.32\%. Compared with the direct encoding of the test set, the compression rate is increased by 11.75\%. 2.47\%.

(2)(2)A method of generating basis vectors based on kmeans ++ clustering algorithm combined with test set is proposed. This method selects the initial clustering center of the test set based on the principle of maximum distance, then calculates the Euclidean distance between each column vector in the test set and the clustering center, and re-divides the column vector into the principle of minimum distance Different clusters and calculate new cluster centers. Iterate iteratively in this way. When the column vectors in the cluster basically no longer change, the resulting cluster center is the base vector sought in this paper. Experimental results show that the use of clustering algorithm combined with split compression method can increase the average compression rate of RL-Huff encoding to 76.30\%. Compared with the direct compression of the test set, the average compression rate is increased by 13.73\%, which is Compared with transcoding, the average compression rate is increased by 4.45\%. In order to further verify the reliability of the method, I conducted related experiments on large circuits such as b15, b17, b20, and b21. Under the FDR coding method, the compression ratio obtained by this method is 6.06\% higher than that of the direct test set.

(3)The kmeans ++ clustering algorithm combined with the bit flip algorithm further improves the compression rate. The main idea of this method is: first, the main component set is used for fault simulation to detect part of the circuit's fault, then the original test set is also used for fault simulation, the main component set can detect the fault corresponding to the corresponding bit is converted into an irrelevant bit, and Generate a new test set. Finally, the new original test set is XORed with the main component set to obtain a new residual component set and improve the compression rate. Experimental results show that the use of bit reversal algorithm combined with kmeans ++ algorithm can increase the average compression rate by 7\% on the basis of (2).
}
\ekeywords{Split compression; basis vector; principal component set; kmeans++ algorithm; test data compression}
\makecover
\clearpage
